\documentclass{article}
\usepackage[top=1cm, bottom=1.5cm, left=1.5cm, right=1.5cm]{geometry}
\usepackage[latin1]{inputenc}
\usepackage[T1]{fontenc}
\usepackage[francais]{babel}
\usepackage{lmodern}
\usepackage{graphicx}

\begin{document}

% d�finition des style de puces
\renewcommand{\labelitemi}{$\bullet$}
\renewcommand{\labelitemii}{$\circ$}
\renewcommand{\labelitemiii}{$-$}
\renewcommand{\labelitemiv}{$\triangleright$}

\title{%
    \begin{minipage}\linewidth
        \centering \bfseries
        Rapport du projet XMLLiteParser
        \vskip3pt
        \large Mod�lisation
    \end{minipage}
	
}

%\title{}
\author{Mathis Deloge, Antoine Petot, Ange Picard}
\date{}
\maketitle
\section{Descriptif du sujet}
\section{Journal de bord}
\subsection{S�ance 1}
\subsection{S�ance 2}
Mod�lisation de l'automate fini. Impl�mentation des diff�rents �tats.
\section{Choix du mod�le math�matique}
\section{Prolongements possibles}
\subsection{Etudiez et justifiez les propri�t�s de la structure math�matique utilis�e.}
\subsection{Modifiez votre validateur afin qu�il permette le d�bugage du fichier XML. Quel impact cette modification a eu sur la structure math�matique utilis�e ?}
\subsection{Modifiez votre validateur afin qu�il s�accorde � un sch�ma pr�d�fini. Quel impact cette modification a eu sur la structure math�matique utilis�e ?}
\subsection{Modifiez votre validateur afin qu�il prenne en compte un sch�ma accompagnant �ventuellement un fichier XML-Lite.}
\subsection{Proposez un sch�ma permettant de stocker la base de donn�es d�un g�n�rateur de QCM, chaque question ayant de 1 � 5 r�ponses, correctes ou non.}
\subsection{Rajoutez � votre programme un interpr�teur (pour le sch�ma du prolongement pr�c�dent).}
\section{Conclusion}

\end{document}