\documentclass{article}
\usepackage[top=1cm, bottom=1.5cm, left=1.5cm, right=1.5cm]{geometry}
\usepackage[latin1]{inputenc}
\usepackage[T1]{fontenc}
\usepackage[francais]{babel}
\usepackage{lmodern}
\usepackage{graphicx}
\usepackage{listings}
\usepackage{color}

\definecolor{mygreen}{rgb}{0,0.6,0}
\definecolor{mygray}{rgb}{0.5,0.5,0.5}
\definecolor{mymauve}{rgb}{0.58,0,0.82}

\lstset{ %
  backgroundcolor=\color{white},   % choose the background color
  basicstyle=\footnotesize,        % size of fonts used for the code
  breaklines=true,                 % automatic line breaking only at whitespace
  captionpos=b,                    % sets the caption-position to bottom
  commentstyle=\color{mygreen},    % comment style
  escapeinside={\%*}{*)},          % if you want to add LaTeX within your code
  keywordstyle=\color{blue},       % keyword style
  stringstyle=\color{mymauve},     % string literal style
}


\title{%
    \begin{minipage}\linewidth
        \centering \bfseries
        Rapport du projet XMLLiteParser
        \vskip3pt
        \large Modélisation
    \end{minipage}
	
}

%\title{}
\author{Mathis Deloge, Antoine Petot, Ange Picard}
\date{}

\begin{document}

% définition des style de puces
\renewcommand{\labelitemi}{$\bullet$}
\renewcommand{\labelitemii}{$\circ$}
\renewcommand{\labelitemiii}{$-$}
\renewcommand{\labelitemiv}{$\triangleright$}


\maketitle
\section{Descriptif du sujet}
\section{Journal de bord}
\subsection{Séance 1}
\subsection{Séance 2}
Modélisation de l'automate fini. Implémentation des différents états.
\section{Choix du modèle mathématique}
\section{Prolongements possibles}
\subsection{Etudiez et justifiez les propriétés de la structure mathématique utilisée.}
\subsection{Modifiez votre validateur afin qu’il permette le débugage du fichier XML. Quel impact cette modification a eu sur la structure mathématique utilisée ?}
\subsection{Modifiez votre validateur afin qu’il s’accorde à un schéma prédéfini. Quel impact cette modification a eu sur la structure mathématique utilisée ?}
\subsection{Modifiez votre validateur afin qu’il prenne en compte un schéma accompagnant éventuellement un fichier XML-Lite.}
\subsection{Proposez un schéma permettant de stocker la base de données d’un générateur de QCM, chaque question ayant de 1 à 5 réponses, correctes ou non.}
\subsection{Rajoutez à votre programme un interpréteur (pour le schéma du prolongement précédent).}
\section{Conclusion}

\section{Comment ajouter du code ?}
\subsection{Comme ça}
\lstinputlisting[language=java, firstline=1, lastline=45]{../src/main.java}

\subsection{Ou comme ça}
\begin{lstlisting}[language=java]
class HelloWorldApp {
    public static void main(String[] args) {
        System.out.println("Hello World!"); // Display the string.
        for (int i = 0; i < 100; ++i) {
            System.out.println(i);
        }
    }
}
\end{lstlisting}

\end{document}